\chapter{利用\LaTeX 排版图片与表格}
在介绍如何排版表格与图片时,先介绍浮动体的概念
\section{浮动体}
在排版中文文档或者实验报告时,尤其是在今后的论文、书籍撰写中,表格与图片均称为\textbf{浮动体}。顾名思义,浮动体在文中的位置不是固定的,美观起见需要自动放在合适的位置,在需要的时候做引用。在排版时,作者需要优先排版文字内容,最后再关注图片位置,不应固定死图片的位置,除了造成大片空白也会使得整体不够美观。
\subsection{浮动体的用法}
一般来说浮动体环境有两种,figure环境与table环境,分别用于浮动图片与表格,用法如下:
\begin{minted}{LaTeX}
\begin{figure}[<placement>]
content...
\end{figure}
\end{minted}

表格与图片用法相同,跟在环境名后面的$placement$提供了浮动体在页面中允许排版的位置,默认为tbp,意为允许在顶部、底部、单独成页排版。
\begin{table}[h!]
\centering
\begin{tabular}{cc}
\hline
h&代码所处的当前位置\\
t&页面顶端\\
b&页码底部\\
p&单独成页\\
!&在决定位置时忽略限制\\
\hline
\end{tabular}
\end{table}
\subsection{浮动体的标题}
在浮动体中,利用\verb|\caption{...}|添加标题,用法与\verb|\chapter{...}|类似,添加的标题会自动编号,figure会在内容前显示如“图 1”的样式,表格类似。

紧跟着\verb|\caption{...}|后面可以添加\verb|\label{key}|命令交叉引用,具体在后面章节叙述。{图片的排版}
\LaTeX 本身不支持插图功能,需要由graphicx 宏包(本模板已添加)辅助支持。在本模板下,可以添加.jpg.pdf.eps.png.bmp格式的图片,

在调用了graphicx包后,可以使用命令\verb|\includegraphics[⟨options⟩]{⟨filename⟩}|插入图片,$filename$是图片的位置,本模板中需要将图片放在figure文件中,并使用相对路径调用,如\verb|figure/filename.png|。$options$是需要的参数,如设置图片宽为0.7$cm$需要在该位置书写\verb|[width=0.7cm]|,具体的参数见表下表。
\begin{table}[h]
\centering
\begin{tabular}{lc}
\hline
参数&含义\\
\hline
width=h&将图片缩放到宽度为h\\
hight=h&将图片缩放到高度为h\\
scale=h&将图片按照原尺寸缩放h倍\\
angle=h&令图片逆时针旋转h度\\
\hline
\end{tabular}
\label{table1}
\end{table}

\begin{minted}{LaTeX}
\begin{figure}[h]
\centering
\includegraphics[scale = 0.7]{example-image-A}
\caption{导入的图片}
\label{fig1}
\end{figure}
\end{minted}
\begin{figure}[h]
\centering
\includegraphics[scale = 0.7]{example-image-A}
\caption{导入的图片}
\label{fig1}
\end{figure}
需要排版并排子图推荐使用subfig宏包,具体使用请看宏包文档{表格的排版}
在实验报告或者论文中,表格是比不可少的部分。下面给出一个简单的表格排版实例:
\begin{minted}{LaTeX}
\begin{table}
\centering
\caption{表格排版实例}
\label{tab1}
\begin{tabular}{|c|l|r|}
\hline
AAA&B&CCC\\
\hline
A&BBB&C\\
\hline
\end{tabular}
\end{table}
\end{minted}

显示为表$^{[\ref{tab1}]}$。
\begin{table}[h]
\centering
\caption{表格排版实例}
\label{tab1}
\begin{tabular}{|c|l|r|}
\hline
AAA&B&CCC\\
\hline
A&BBB&C\\
\hline
\end{tabular}
\end{table}
与多行公式类似,表格排版中的列由tabular环境后的参数决定,c、l、r分别代表居中、居左、居右对齐,必须与列数相同,参数之间的竖线代表是否在表格中绘制竖线。行之间需要添加横线需要命令\verb|\hline|,如果需要合并单元格或者其他操作,具体见lshort表格排版章节,这里只讲述简单的部分。三线表的绘制只需要将参数中的竖线去除即可。

对于初次使用者而言,表格排版是一个很大的难题,在excel中有一个类似的插件可以快捷的生成大致的表格,在打开excel加载项后,下载插件\verb|excel2latex|即可。在生成大致表格后进行细微的调整,可以快速的绘制出想要的表格。